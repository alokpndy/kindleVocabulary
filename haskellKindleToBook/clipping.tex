\documentclass{book}\usepackage[utf8]{inputenc}\title{Kindle Highlights}\begin{document}\maketitle{}{\LARGE{}If you want to know what a man’s like, take a good look at how he treats his inferiors, not his equals.”
}\begin{flushright}by J.K. Rowling, 4: Harry Potter and the Goblet of Fire \end{flushright}

\begin{center}-----------------\end{center}

\maketitle{}{\LARGE{}Where your treasure is, there will your heart be also.
}\begin{flushright}by J.K. Rowling, 
7: Harry Potter and the Deathly Hallows \end{flushright}

\begin{center}-----------------\end{center}

\maketitle{}{\LARGE{}“We was told he might come in here!” said Carrow. “I dunno why, do I?”
}\begin{flushright}by J.K. Rowling, 
7: Harry Potter and the Deathly Hallows \end{flushright}

\begin{center}-----------------\end{center}

\maketitle{}{\LARGE{}To tell the truth, I didn’t know what might happen. I thought the chances were 50-50 that I would get caught. But I am “a cockeyed optimist,”
}\begin{flushright}by Stephen King, 
Mr. Mercedes \end{flushright}

\begin{center}-----------------\end{center}

\maketitle{}{\LARGE{}and I prepared for Success rather than Failure.
}\begin{flushright}by Stephen King, 
Mr. Mercedes \end{flushright}

\begin{center}-----------------\end{center}

\maketitle{}{\LARGE{}Well if it had been right there, if Mrs. Trelawney’s Mercedes had stalled or something (small chance of that as it seemed very well maintained), I suppose the crowd might have torn me apart,
}\begin{flushright}by Stephen King, 
Mr. Mercedes \end{flushright}

\begin{center}-----------------\end{center}

\maketitle{}{\LARGE{}He gropes
}\begin{flushright}by Stephen King, 
Mr. Mercedes \end{flushright}

\begin{center}-----------------\end{center}

\maketitle{}{\LARGE{}incremental repetition,
}\begin{flushright}by Stephen King, 
Mr. Mercedes \end{flushright}

\begin{center}-----------------\end{center}

\maketitle{}{\LARGE{}rich people tend to be slightly more
}\begin{flushright}by Stephen King, 
Mr. Mercedes \end{flushright}

\begin{center}-----------------\end{center}

\maketitle{}{\LARGE{}Nietzsche
}\begin{flushright}by Stephen King, 
Mr. Mercedes \end{flushright}

\begin{center}-----------------\end{center}

\maketitle{}{\LARGE{}You can lead a whore to culture, but you can’t make her think.”
}\begin{flushright}by Stephen King, 
Mr. Mercedes \end{flushright}

\begin{center}-----------------\end{center}

\maketitle{}{\LARGE{}Sometimes those who love her say bad things of her but they are always said as though she were a woman. Some of the younger fishermen, those who used buoys as floats for their lines and had motorboats, bought when the shark livers had brought much money, spoke of her as el mar which is masculine.
}\begin{flushright}by Ernest Hemingway, 
The Old Man and the Sea \end{flushright}

\begin{center}-----------------\end{center}

\maketitle{}{\LARGE{}forty fathoms.
}\begin{flushright}by Ernest Hemingway, 
The Old Man and the Sea \end{flushright}

\begin{center}-----------------\end{center}

\maketitle{}{\LARGE{}because he knew that if you said a good thing it might not happen.
}\begin{flushright}by Ernest Hemingway, 
The Old Man and the Sea \end{flushright}

\begin{center}-----------------\end{center}

\maketitle{}{\LARGE{}“Take a good rest, small bird,” he said. “Then go in and take your chance like any man or bird or fish.”
}\begin{flushright}by Ernest Hemingway, 
The Old Man and the Sea \end{flushright}

\begin{center}-----------------\end{center}

\maketitle{}{\LARGE{}He is a great fish and I must convince him, he thought. I must never let him learn his strength nor what he could do if he made his run.
}\begin{flushright}by Ernest Hemingway, 
The Old Man and the Sea \end{flushright}

\begin{center}-----------------\end{center}

\maketitle{}{\LARGE{}If I were him I would put in everything now and go until something broke. But, thank God, they are not as intelligent as we who kill them; although they are more noble and more able.
}\begin{flushright}by Ernest Hemingway, 
The Old Man and the Sea \end{flushright}

\begin{center}-----------------\end{center}

\maketitle{}{\LARGE{}Design changes don’t always create the result you intended and sometimes have the opposite effect of what you expected.
}\begin{flushright}by Lukas Mathis, 
Designed for Use \end{flushright}

\begin{center}-----------------\end{center}

\maketitle{}{\LARGE{}But people will still surprise you by finding creative ways of misunderstanding your application’s user interface,
}\begin{flushright}by Lukas Mathis, 
Designed for Use \end{flushright}

\begin{center}-----------------\end{center}

\maketitle{}{\LARGE{}illogical ways,
}\begin{flushright}by Lukas Mathis, 
Designed for Use \end{flushright}

\begin{center}-----------------\end{center}

\maketitle{}{\LARGE{}Use common sense when designing user interfaces, but don’t rely on it.
}\begin{flushright}by Lukas Mathis, 
Designed for Use \end{flushright}

\begin{center}-----------------\end{center}

\maketitle{}{\LARGE{}typically, design processes are iterative.
}\begin{flushright}by Lukas Mathis, 
Designed for Use \end{flushright}

\begin{center}-----------------\end{center}

\maketitle{}{\LARGE{}about the people who are going to use their product
}\begin{flushright}by Lukas Mathis, 
Designed for Use \end{flushright}

\begin{center}-----------------\end{center}

\maketitle{}{\LARGE{}Rather than asking for something different that actually fixes their problems, they ask for the same thing that’s slightly better.
}\begin{flushright}by Lukas Mathis, 
Designed for Use \end{flushright}

\begin{center}-----------------\end{center}

\maketitle{}{\LARGE{}people may not even be able to tell us what their problems
}\begin{flushright}by Lukas Mathis, 
Designed for Use \end{flushright}

\begin{center}-----------------\end{center}

\maketitle{}{\LARGE{}people are pretty bad at predicting whether and how they would use a product if we proposed to build it for them.
}\begin{flushright}by Lukas Mathis, 
Designed for Use \end{flushright}

\begin{center}-----------------\end{center}

\maketitle{}{\LARGE{}never trust focus groups.
}\begin{flushright}by Lukas Mathis, 
Designed for Use \end{flushright}

\begin{center}-----------------\end{center}

\maketitle{}{\LARGE{}Heaven protect the man who tried to have a conversation with him about politics or philosophy. A ten-year-old child would know more.
}\begin{flushright}by Horowitz, Anthony, 
The House of Silk: The New Sherlock Holmes Novel \end{flushright}

\begin{center}-----------------\end{center}

\maketitle{}{\LARGE{}Show Holmes a drop of water and he would deduce the existence of the Atlantic. Show it to me and I would look for a tap. That was the difference between us.
}\begin{flushright}by Horowitz, Anthony, 
The House of Silk: The New Sherlock Holmes Novel \end{flushright}

\begin{center}-----------------\end{center}

\maketitle{}{\LARGE{}‘All emotions and that one (love) particularly, were abhorrent to his cold, precise but admirably balanced mind.’
}\begin{flushright}by Horowitz, Anthony, 
The House of Silk: The New Sherlock Holmes Novel \end{flushright}

\begin{center}-----------------\end{center}

\maketitle{}{\LARGE{}have to say that I plucked quite a few words out of the original stories to act as guideposts, to give the text a sense of authenticity. My favourites are: ‘snibbed’, ‘foeman’, ‘sickish’ (used by Lestrade) and ‘passementerie’. That said, the book is actually being written in around 1916 and I would imagine that by this time Watson’s own language and writing style would have become more modern.
}\begin{flushright}by Horowitz, Anthony, 
The House of Silk: The New Sherlock Holmes Novel \end{flushright}

\begin{center}-----------------\end{center}

\maketitle{}{\LARGE{}Include all the best-known characters – but try and do so in a way that will surprise. Mrs Hudson is there, of course, as well as Lestrade, Mycroft and Wiggins.
}\begin{flushright}by Horowitz, Anthony, 
The House of Silk: The New Sherlock Holmes Novel \end{flushright}

\begin{center}-----------------\end{center}

\maketitle{}{\LARGE{}August, 2011, Crete
}\begin{flushright}by Horowitz, Anthony, 
The House of Silk: The New Sherlock Holmes Novel \end{flushright}

\begin{center}-----------------\end{center}

\maketitle{}{\LARGE{}canoe is being careened on the shore. Henry \& I struck
}\begin{flushright}by David Stephen Mitchell, 
Cloud Atlas: A Novel \end{flushright}

\begin{center}-----------------\end{center}

\maketitle{}{\LARGE{}making them go away usually seems to fix the problem.
}\begin{flushright}by Lukas Mathis, 
Designed for Use \end{flushright}

\begin{center}-----------------\end{center}

\maketitle{}{\LARGE{}Instead, allow them to undo their change.
}\begin{flushright}by Lukas Mathis, 
Designed for Use \end{flushright}

\begin{center}-----------------\end{center}

\maketitle{}{\LARGE{}Similarly, if an error occurs and you have a way to make your product recover on its own without telling the user, do it. If the user has entered a website address that is truncated but your website receives enough information to identify the page he’s looking for, simply forward him to that page. If your application tries to connect to a server but the connection times out, make the application try again before telling the user there’s something wrong. Notify the user only if your product really can’t fix the problem on its own.
}\begin{flushright}by Lukas Mathis, 
Designed for Use \end{flushright}

\begin{center}-----------------\end{center}

\maketitle{}{\LARGE{}use verbs as button labels,
}\begin{flushright}by Lukas Mathis, 
Designed for Use \end{flushright}

\begin{center}-----------------\end{center}

\maketitle{}{\LARGE{}“Writers often believe that they should communicate more than readers want to know.”
}\begin{flushright}by Lukas Mathis, 
Designed for Use \end{flushright}

\begin{center}-----------------\end{center}

\maketitle{}{\LARGE{}To avoid confusing them, ask yourself whether a sentence is unambiguous even if you’ve read only part of it.
}\begin{flushright}by Lukas Mathis, 
Designed for Use \end{flushright}

\begin{center}-----------------\end{center}

\maketitle{}{\LARGE{}“No one will ever complain because you have made something too easy to understand.”
}\begin{flushright}by Lukas Mathis, 
Designed for Use \end{flushright}

\begin{center}-----------------\end{center}

\maketitle{}{\LARGE{}Cloze test.
}\begin{flushright}by Lukas Mathis, 
Designed for Use \end{flushright}

\begin{center}-----------------\end{center}

\maketitle{}{\LARGE{}Buttons affect only things that are on the same or on a lower hierarchical level.
}\begin{flushright}by Lukas Mathis, 
Designed for Use \end{flushright}

\begin{center}-----------------\end{center}

\maketitle{}{\LARGE{}I, your father, and you
}\begin{flushright}by David Mitchell, 
The Thousand Autumns of Jacob de Zoet \end{flushright}

\begin{center}-----------------\end{center}

\maketitle{}{\LARGE{}I, your father, and you and Geertje owe
}\begin{flushright}by David Mitchell, 
The Thousand Autumns of Jacob de Zoet \end{flushright}

\begin{center}-----------------\end{center}

\maketitle{}{\LARGE{}is a gift from your ancestors and a loan from your descendants.
}\begin{flushright}by David Mitchell, 
The Thousand Autumns of Jacob de Zoet \end{flushright}

\begin{center}-----------------\end{center}

\maketitle{}{\LARGE{}Preserved from decay by alcohol, pig bladder, and lead, they warn not so much that all flesh perishes—what sane adult forgets this truth for long?—but that immortality comes at a steep price.
}\begin{flushright}by David Mitchell, 
The Thousand Autumns of Jacob de Zoet \end{flushright}

\begin{center}-----------------\end{center}

\maketitle{}{\LARGE{}He is pink with heat and shiny with sweat.
}\begin{flushright}by David Mitchell, 
The Thousand Autumns of Jacob de Zoet \end{flushright}

\begin{center}-----------------\end{center}

\maketitle{}{\LARGE{}shall have Twomey fashion me one of those ingenious cloth fans the English brought from India … oh, the word evades me …” “Might you be thinking of a punkah, sir?”
}\begin{flushright}by David Mitchell, 
The Thousand Autumns of Jacob de Zoet \end{flushright}

\begin{center}-----------------\end{center}

\maketitle{}{\LARGE{}“Asiatic minds respect force majeure;
}\begin{flushright}by David Mitchell, 
The Thousand Autumns of Jacob de Zoet \end{flushright}

\begin{center}-----------------\end{center}

\maketitle{}{\LARGE{}Hawkers cry, beggars implore, tinkers clang pans,
}\begin{flushright}by David Mitchell, 
The Thousand Autumns of Jacob de Zoet \end{flushright}

\begin{center}-----------------\end{center}

\maketitle{}{\LARGE{}Children on a mud wall make owl eyes with their forefingers and thumbs,
}\begin{flushright}by David Mitchell, 
The Thousand Autumns of Jacob de Zoet \end{flushright}

\begin{center}-----------------\end{center}

\maketitle{}{\LARGE{}Jacob realizes they are impersonating “round” European eyes and remembers a string of urchins following a Chinaman in London. The urchins pulled their eyes into narrow slants and sang, “Chinese, Siamese, if you please, Japanese.”
}\begin{flushright}by David Mitchell, 
The Thousand Autumns of Jacob de Zoet \end{flushright}

\begin{center}-----------------\end{center}

\maketitle{}{\LARGE{}The bitter dregs make him wince and amplify his headache but sharpen his wits.
}\begin{flushright}by David Mitchell, 
The Thousand Autumns of Jacob de Zoet \end{flushright}

\begin{center}-----------------\end{center}

\maketitle{}{\LARGE{}“A diet of abstinence,” replies Vorstenbosch, “never hurt anyone.”
}\begin{flushright}by David Mitchell, 
The Thousand Autumns of Jacob de Zoet \end{flushright}

\begin{center}-----------------\end{center}

\maketitle{}{\LARGE{}Jacob gets up, drinks from a cracked jug, and rubs soap into lather.
}\begin{flushright}by David Mitchell, 
The Thousand Autumns of Jacob de Zoet \end{flushright}

\begin{center}-----------------\end{center}

\maketitle{}{\LARGE{}In Van Cleef’s upper window, the deputy’s latest “wife” combs her hair. She smiles at Jacob; Melchior van Cleef, his chest hairy as a bear’s, appears. “‘Thou shalt not,’” he quotes, “‘dip thy nib in another man’s inkpot.’”
}\begin{flushright}by David Mitchell, 
The Thousand Autumns of Jacob de Zoet \end{flushright}

\begin{center}-----------------\end{center}

\maketitle{}{\LARGE{}The gods depended on them for their strength; and the arya depended on the strength
}\begin{flushright}by Keay, John, 
India: A History. Revised and Updated \end{flushright}

\begin{center}-----------------\end{center}

\maketitle{}{\LARGE{}the Daksinapatha (whence the term ‘Deccan’) or
}\begin{flushright}by Keay, John, 
India: A History. Revised and Updated \end{flushright}

\begin{center}-----------------\end{center}

\maketitle{}{\LARGE{}“Never believe anything you hear at a woman’s tit.
}\begin{flushright}by Martin, George R.R., 
A Game of Thrones: A Song of Ice and Fire: Book One \end{flushright}

\begin{center}-----------------\end{center}

\maketitle{}{\LARGE{}soft as sin.
}\begin{flushright}by Martin, George R.R., 
A Game of Thrones: A Song of Ice and Fire: Book One \end{flushright}

\begin{center}-----------------\end{center}

\maketitle{}{\LARGE{}matsya-nyaya (‘the law of the fishes’, i.e. of the jungle);
}\begin{flushright}by Keay, John, 
India: A History. Revised and Updated \end{flushright}

\begin{center}-----------------\end{center}

\maketitle{}{\LARGE{}The gods, or Lord Vishnu in the shape of that rapidly growing fish, proposed a raja ; and they selected Manu.
}\begin{flushright}by Keay, John, 
India: A History. Revised and Updated \end{flushright}

\begin{center}-----------------\end{center}

\maketitle{}{\LARGE{}vaisya, continued as gramini and grhpati, villagers and household heads. Their role was that of creating the wealth on which the ksatriya and brahman depended or, as the texts have it, on which ksatriya and brahman might ‘graze’.
}\begin{flushright}by Keay, John, 
India: A History. Revised and Updated \end{flushright}

\begin{center}-----------------\end{center}

\maketitle{}{\LARGE{}Much later, just as the ksatriya in recognition of their martial status would be equated with ‘rajputs’, so the vaisya would be identified with the essentially mercantile ‘bania’.
}\begin{flushright}by Keay, John, 
India: A History. Revised and Updated \end{flushright}

\begin{center}-----------------\end{center}

\maketitle{}{\LARGE{}Just as the vaisya was expected to furnish wealth, the sudra was expected to furnish labour.
}\begin{flushright}by Keay, John, 
India: A History. Revised and Updated \end{flushright}

\begin{center}-----------------\end{center}

\maketitle{}{\LARGE{}These then were the four earliest castes, and a much-quoted passage from the latest mandala (X) of the Rig Veda clearly shows their relative status.
}\begin{flushright}by Keay, John, 
India: A History. Revised and Updated \end{flushright}

\begin{center}-----------------\end{center}

\maketitle{}{\LARGE{}brahman was his mouth, of both arms was the rajanya (ksatriya) made, his thighs became the vaisya, from his feet the sudra was produced.’
}\begin{flushright}by Keay, John, 
India: A History. Revised and Updated \end{flushright}

\begin{center}-----------------\end{center}

\maketitle{}{\LARGE{}The term used for caste in the Vedas is varna, ‘colour’, which, in the context of the arya’s disparaging comments about the ‘black’dasa, is often taken to mean that the higher castes also considered themselves the fairer-skinned
}\begin{flushright}by Keay, John, 
India: A History. Revised and Updated \end{flushright}

\begin{center}-----------------\end{center}

\maketitle{}{\LARGE{}ware, its acceptance from one end of northern India to the other hinted at a social, cultural and linguistic cohesion which belied the multiplicity of states and could – indeed imminently would – transcend them.
}\begin{flushright}by Keay, John, 
India: A History. Revised and Updated \end{flushright}

\begin{center}-----------------\end{center}

\maketitle{}{\LARGE{}Since neither Mahavira nor the Buddha ventured south, their followers had little to record of the area and there are no textual sources for it before the end of the first millennium BC.
}\begin{flushright}by Keay, John, 
India: A History. Revised and Updated \end{flushright}

\begin{center}-----------------\end{center}

\maketitle{}{\LARGE{}In fact to this day indigenous vaisya and ksatriya castes are practically unknown in peninsular India.
}\begin{flushright}by Keay, John, 
India: A History. Revised and Updated \end{flushright}

\begin{center}-----------------\end{center}

\maketitle{}{\LARGE{}The Indus, to which most of these seven rivers were tributary, was the sindhu par excellence ; and in the language of ancient Persian, a near relative of Sanskrit, the initial ‘s’ of a Sanskrit word was invariably rendered as an aspirate – ‘h’. Soma, the mysterious hallucinogen distilled, deified and drunk to excess by the Vedic arya, is thus homa or haoma in old Persian; and sindhu is thus Hind[h]u. When, from Persian, the word found its way into Greek, the initial aspirate was dropped, and it started to appear as the route ‘Ind’ (as in ‘India’, ‘Indus’, etc.).
}\begin{flushright}by Keay, John, 
India: A History. Revised and Updated \end{flushright}

\begin{center}-----------------\end{center}

\maketitle{}{\LARGE{}was also in Taxila that, in the previous century, Panini compiled a grammar more comprehensive and scientific than any dreamed of by Greek grammarians.
}\begin{flushright}by Keay, John, 
India: A History. Revised and Updated \end{flushright}

\begin{center}-----------------\end{center}

\end{document}